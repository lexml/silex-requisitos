\newglossaryentry{ab-rogacao}{
       name={Ab-rogação},
       description={ \textit{Ver:} \Gls{revogacao-total-da-norma}.},
      }
\newglossaryentry{acao-declaratoria-de-constitucionalidade}{
       name={Ação Declaratória de Constitucionalidade},
       description={Ação que tem por finalidade confirmar que uma lei ou parte dela é constitucional.
\newline \textit{Acrônimo(s):} ADC.
\newline \textit{Específico de:} \Gls{norma-juridica}.},
      }
\newglossaryentry{acao-direta-de-inconstitucionalidade}{
       name={Ação Direta de Inconstitucionalidade},
       description={Ação que tem por finalidade declarar que uma lei ou parte dela é inconstitucional.
\newline \textit{Acrônimo(s):} ADI ou ADIn.
\newline \textit{Específico de:} \Gls{norma-juridica}.},
      }
\newglossaryentry{acrescimo}{
       name={Acréscimo},
       description={Espécie de modificação que adiciona dispositivos ou expressões à norma jurídica.
\newline \textit{Específico de:} \Gls{modificacao}.
\newline \textit{Geral de:} \Gls{acrescimo-de-dispositivo}, \Gls{acrescimo-de-expressao}.},
      }
\newglossaryentry{acrescimo-de-dispositivo}{
       name={Acréscimo de Dispositivo},
       description={Espécie de modificação que adiciona dispositivos à norma jurídica.
\newline \textit{Específico de:} \Gls{acrescimo}.},
      }
\newglossaryentry{acrescimo-de-expressao}{
       name={Acréscimo de Expressão},
       description={Espécie de modificação que adiciona expressões à norma jurídica.
\newline \textit{Específico de:} \Gls{acrescimo}.},
      }
\newglossaryentry{alerta}{
       name={Alerta},
       description={Relacionamento genérico para expressar notas que auxiliem na interpretação e aplicação das normas. Pode ser utilizado para alertar sobre atualização de valores monetários, para mudanças genéricas de expressões ou para destacar uma ressalva de uma revogação.
\newline \textit{Ver também:} \Gls{revogacao-com-ressalva}, \Gls{revogacao-de-norma-integral-com-ressalva}, \Gls{revogacao-tacita}.},
      }
\newglossaryentry{alerta--no-documento}{
       name={Alerta (no Documento)},
       description={ \textit{Ver:} \Gls{nota}.},
      }
\newglossaryentry{alteracao}{
       name={Alteração},
       description={Texto de norma jurídica que estabelece disposições gerais ou especiais a diploma legal.
\newline \textit{Específico de:} \Gls{modificacao}.
\newline \textit{Geral de:} \Gls{alteracao-expressa}, \Gls{alteracao-tacita}.
\newline \textit{Ver também:} \Gls{ressalva-de-aplicacao}, \Gls{retificacao--com-sentido-de-alteracao}, \Gls{texto-compilado}.},
      }
\newglossaryentry{alteracao-expressa}{
       name={Alteração Expressa},
       description={É a alteração diretamente determinada em diploma legal.
\newline \textit{Específico de:} \Gls{alteracao}.
\newline \textit{Geral de:} \Gls{alteracao-expressa-de-dispositivo}, \Gls{alteracao-expressa-de-expressao}, \Gls{supressao-de-expressao}.},
      }
\newglossaryentry{alteracao-expressa-de-dispositivo}{
       name={Alteração Expressa de Dispositivo},
       description={É a alteração expressa que dá nova redação a todo um dispositivo.
\newline \textit{Específico de:} \Gls{alteracao-expressa}.},
      }
\newglossaryentry{alteracao-expressa-de-expressao}{
       name={Alteração Expressa de Expressão},
       description={É a alteração expressa que substitui uma expressão no dispositivo.
\newline \textit{Específico de:} \Gls{alteracao-expressa}.},
      }
\newglossaryentry{alteracao-tacita}{
       name={Alteração Tácita},
       description={Evento em que a norma anterior é alterada se há incompatibilidade entre ela e os preceitos da nova norma. O SILEX não registra os relacionamentos de alteração tácita.
\newline \textit{Específico de:} \Gls{alteracao}.
\newline \textit{Ver também:} \Gls{novo-tratamento-da-materia}.},
      }
\newglossaryentry{anulacao}{
       name={Anulação},
       description={Evento que retira do mundo jurídico atos com defeito de validade (atos inválidos), produzindo efeitos retroativos à data em que o ato foi emitido (efeitos ex tunc). Excepcionalmente, no âmbito das normas infralegais, o termo anulação pode ser utilizado na acepção de revogação.
\newline \textit{Ver também:} \Gls{ato-invalido}, \Gls{revogacao}.},
      }
\newglossaryentry{apreciacao-de-veto}{
       name={Apreciação de Veto},
       description={É a manifestação de competência exclusiva do Poder Legislativo sobre veto aposto a Projeto de Lei pelo Chefe do Poder Executivo. No âmbito do SILEX, apenas será considerado o resultado da apreciação que rejeita o veto.
\newline \textit{Ver também:} \Gls{publicacao-de-veto-rejeitado}, \Gls{rejeicao-do-veto}.},
      }
\newglossaryentry{aprovacao-de-ato-internacional}{
       name={Aprovação de Ato Internacional},
       description={Ato pelo qual um decreto legislativo aprova o texto de um ato internacional autorizando o Executivo a ratificar o ato no plano internacional.},
      }
\newglossaryentry{arguicao-de-descumprimento-de-preceito-fundamental}{
       name={Arguição de Descumprimento de Preceito Fundamental},
       description={Ação que tem por finalidade evitar ou reparar lesão a preceito fundamental da Constituição, resultante de ato do Poder Público. Entretanto, esse tipo de ação também pode ter natureza equivalente às ADIs, podendo questionar a constitucionalidade de uma norma perante a Constituição Federal, mas tal norma deve ser municipal ou anterior à Constituição vigente (no caso, anterior à de 1988). A ADPF é disciplinada pela Lei Federal 9.882/1999.
\newline \textit{Acrônimo(s):} ADPF.
\newline \textit{Específico de:} \Gls{norma-juridica}.},
      }
\newglossaryentry{ato-cancelado}{
       name={Ato Cancelado},
       description={ \textit{Ver:} \Gls{ato-invalido}.},
      }
\newglossaryentry{ato-invalido}{
       name={Ato Inválido},
       description={Na normatização infralegal, atos administrativos praticados em desconformidade com as prescrições jurídicas são passíveis anulação. Excepcionalmente, o termo pode ser utilizado na acepção de revogação.
\newline \textit{Ver também:} \Gls{anulacao}, \Gls{revogacao}.},
      }
\newglossaryentry{ato-nulo}{
       name={Ato Nulo},
       description={ \textit{Ver:} \Gls{ato-invalido}.},
      }
\newglossaryentry{caducidade-expressa}{
       name={Caducidade Expressa},
       description={Condição jurídica da norma que, com prazo certo de vigência, perde sua eficácia por falta de prática de determinadas formalidades (Ref. Legislativa: art. 26, § 6º, arts. 31 e 32, Decreto-Lei 227/1967).
\newline \textit{Ver também:} \Gls{declaracao-de-caducidade}, \Gls{perempcao-de-concessao-ou-permissao}.},
      }
\newglossaryentry{caducidade-tacita}{
       name={Caducidade Tácita},
       description={Atos administrativos normativos podem ser considerados caducos por sua não aplicação, desuso, perda de oportunidade ou de sentido, em face da mudança de hábitos, costumes ou mesmo da situação que alguns deles pretendiam regulamentar.},
      }
\newglossaryentry{caduco}{
       name={Caduco},
       description={ \textit{Ver:} \Gls{caducidade-expressa}.},
      }
\newglossaryentry{citacao-legislativa}{
       name={Citação Legislativa},
       description={ \textit{Ver:} \Gls{remissao}.},
      }
\newglossaryentry{clausula-de-vigencia}{
       name={Cláusula de Vigência},
       description={Dispositivo de Norma Jurídica que estabelece o início ou o final do período de vigência.
\newline \textit{Geral de:} \Gls{clausula-de-vigencia--interna}, \Gls{clausula-de-vigencia--externa}.},
      }
\newglossaryentry{clausula-de-vigencia--externa}{
       name={Cláusula de Vigência (Externa)},
       description={Dispositivo de Norma Jurídica que estabelece o início ou o final do período de vigência de outra Norma Jurídica.
\newline \textit{Específico de:} \Gls{clausula-de-vigencia}.},
      }
\newglossaryentry{clausula-de-vigencia--interna}{
       name={Cláusula de Vigência (Interna)},
       description={Dispositivo da própria Norma Jurídica que estabelece o início ou o final do período de vigência.
\newline \textit{Específico de:} \Gls{clausula-de-vigencia}.},
      }
\newglossaryentry{compilacao}{
       name={Compilação},
       description={Consiste na integração das alterações ocorridas durante a vigência do diploma legal. Tem por finalidade abreviar e facilitar a consulta em todas as fontes de informação legislativa. Poderá gerar um texto multivigente ou um texto atualizado para uma determinada data.
\newline \textit{Ver também:} \Gls{texto-compilado}.},
      }
\newglossaryentry{consolidacao}{
       name={Consolidação},
       description={Consiste na integração de todas as leis pertinentes a determinada matéria num único diploma legal, revogando-se formalmente as leis incorporadas à consolidação, sem modificação do alcance nem interrupção da força normativa dos dispositivos consolidados. (Parágrafo 1º. do art. 13 da Lei Complementar nº 95 com redação dada pela Lei Complementar nº 107, de 26.4.2001).
\newline \textit{Nota:} Está fora do escopo do SILEX.},
      }
\newglossaryentry{convalidacao}{
       name={Convalidação},
       description={Correção ou ratificação de um ato normativo eivado de vícios, tornando-o válido e perfeito.},
      }
\newglossaryentry{conversao}{
       name={Conversão},
       description={Aprovação de medida provisória e sua transformação em lei.
\newline \textit{Geral de:} \Gls{conversao-com-alteracao}, \Gls{conversao-sem-alteracao}.},
      }
\newglossaryentry{conversao-com-alteracao}{
       name={Conversão com Alteração},
       description={Aprovação de medida provisória com alteração de texto e sua transformação em lei.
\newline \textit{Específico de:} \Gls{conversao}.},
      }
\newglossaryentry{conversao-de-decreto-lei-em-medida-provisoria}{
       name={Conversão de Decreto-Lei em Medida Provisória},
       description={Os decretos-leis editados entre 3 de setembro de 1988 e a promulgação da Constituição de 1988 foram convertidos em Medidas Provisórias (Referência Legislativa: CF, 1988, ADCT, art. 25).},
      }
\newglossaryentry{conversao-sem-alteracao}{
       name={Conversão sem Alteração},
       description={Aprovação de medida provisória sem alteração de texto e sua transformação em lei.
\newline \textit{Específico de:} \Gls{conversao}.},
      }
\newglossaryentry{correlacao}{
       name={Correlação},
       description={Associação entre entidades cujo tema tem relação entre si. Consideram-se, entre outros casos, as relações entre normas jurídicas, entre jurisprudência e normas jurídicas e entre jurisprudência e doutrina.},
      }
\newglossaryentry{decisao-liminar-em-sede-de-adi}{
       name={Decisão Liminar em Sede de ADI},
       description={Decisão judicial de caráter provisório que, entre outras coisas, pode suspender a eficácia de determinada norma.
\newline \textit{Ver também:} \Gls{declaracao-de-inconstitucionalidade}, \Gls{suspensao-de-eficacia}.},
      }
\newglossaryentry{declaracao-de-caducidade}{
       name={Declaração de Caducidade},
       description={Ato pelo qual a autoridade competente reconhece a situação de caducidade de determinada norma.
\newline \textit{Ver também:} \Gls{caducidade-expressa}, \Gls{declaracao-de-perempcao-de-concessao-ou-permissao}.},
      }
\newglossaryentry{declaracao-de-eficacia-expressa}{
       name={Declaração de Eficácia Expressa},
       description={Dispositivo pelo qual a autoridade competente estabelece eficácia retroativa ou adiada de forma explícita.},
      }
\newglossaryentry{declaracao-de-inconstitucionalidade}{
       name={Declaração de Inconstitucionalidade},
       description={Ato pelo qual se declara inconstitucional uma norma jurídica no todo ou em parte, mediante o ajuizamento de uma ação direta de inconstitucionalidade ou ação declaratória de constitucionalidade perante o tribunal constitucional. Pode ocorrer efeito repristinatório, ocasião em que as versões das normas afetadas voltam ao seu estado anterior. (Vide LCP 95/1998, Art. 12, III, c -- veda o aproveitamento do numero do dispositivo declarado inconstitucional).
\newline \textit{Ver também:} \Gls{decisao-liminar-em-sede-de-adi}, \Gls{efeito-repristinatorio}.},
      }
\newglossaryentry{declaracao-de-insubsistencia}{
       name={Declaração de Insubsistência},
       description={ \textit{Ver:} \Gls{anulacao}.},
      }
\newglossaryentry{declaracao-de-insubsistencia-de-medida-provisoria}{
       name={Declaração de Insubsistência de Medida Provisória},
       description={Ato pelo qual o Chefe do Poder Legislativo declara insubsistente a Medida Provisória que foi rejeitada ou arquivada pela Casa Legislativa, feita a devida comunicação ao Chefe do Poder Executivo. Após a Resolução CN 1/2002, passou a ser utilizado o termo ``Rejeição de Medida Provisória'' (vide: Resolução CN 1/1989)
\newline \textit{Ver também:} \Gls{rejeicao-de-medida-provisoria}.},
      }
\newglossaryentry{declaracao-de-nao-recepcao-pela-constituicao}{
       name={Declaração de Não Recepção pela Constituição},
       description={Ato pelo qual se declara que uma norma jurídica no todo ou em parte não foi recepcionada pela constituição vigente, mediante o ajuizamento de uma arguição de descumprimento de preceito fundamental (ADPF) perante o tribunal constitucional.},
      }
\newglossaryentry{declaracao-de-perempcao-de-concessao-ou-permissao}{
       name={Declaração de Perempção de Concessão ou Permissão},
       description={Ato pelo qual se declara a perempção de concessão ou permissão.
\newline \textit{Ver também:} \Gls{declaracao-de-caducidade}, \Gls{perempcao-de-concessao-ou-permissao}.},
      }
\newglossaryentry{declaracao-de-prejudicialidade-de-medida-provisoria}{
       name={Declaração de Prejudicialidade de Medida Provisória},
       description={Situação que encerra a vigência da medida provisória por edição de norma que trata sobre a mesma matéria comunicada ao Poder Executivo pelo Poder Legislativo.},
      }
\newglossaryentry{decurso-de-prazo}{
       name={Decurso de Prazo},
       description={Evento que encerra um prazo expressamente estabelecido.},
      }
\newglossaryentry{derrogacao}{
       name={Derrogação},
       description={ \textit{Ver:} \Gls{revogacao-parcial}.},
      }
\newglossaryentry{derrubada-de-veto}{
       name={Derrubada de Veto},
       description={ \textit{Ver:} \Gls{rejeicao-do-veto}.},
      }
\newglossaryentry{disciplinamento-de-relacoes-juridicas-decorrentes-de-medidas-provisorias-rejeitadas-ou-declaradas-insubsistentes}{
       name={Disciplinamento de Relações Jurídicas de MPV Rejeitada ou Declarada Insubsistente},
       description={Norma veiculada por Decreto Legislativo que regula as relações jurídicas derivadas de medidas provisórias rejeitadas ou declaradas insubsistentes.},
      }
\newglossaryentry{dispositivo-conexo}{
       name={Dispositivo Conexo},
       description={ \textit{Ver:} \Gls{correlacao}.},
      }
\newglossaryentry{dispositivo-de-norma-juridica}{
       name={Dispositivo de Norma Jurídica},
       description={Unidade de articulação de um texto normativo. Além da unidade básica (artigo), são considerados dispositivos os desdobramentos de artigo (caput, parágrafos, incisos, alíneas e itens) e os agrupadores de artigo (parte, livro, título, capítulo, seção e subseção).
\newline \textit{Específico de:} \Gls{norma-juridica}.},
      }
\newglossaryentry{dispositivo-de-proposicao-legislativa}{
       name={Dispositivo de Proposição Legislativa},
       description={Unidade de articulação do texto de uma proposição legislativa. Além da unidade básica (artigo), são considerados dispositivos os desdobramentos de artigo (caput, parágrafos, incisos, alíneas e itens) e os agrupadores de artigo (parte, livro, título, capítulo, seção e subseção). Está fora do escopo do SILEX.
\newline \textit{Ver também:} \Gls{proposicao-legislativa}.},
      }
\newglossaryentry{documento}{
       name={Documento},
       description={Registro de uma informação independentemente da natureza do suporte que a contém. Inclui norma, acórdão, súmula, proposição legislativa etc.},
      }
\newglossaryentry{efeito-repristinatorio}{
       name={Efeito Repristinatório},
       description={A manutenção da vigência de norma ou dispositivo aparentemente revogado por outra norma declarada nula em sede de ADIN, em decorrência do princípio da nulidade do ato inconstitucional.
\newline \textit{Ver também:} \Gls{declaracao-de-inconstitucionalidade}.},
      }
\newglossaryentry{eficacia}{
       name={Eficácia},
       description={Produção dos efeitos jurídicos de um ato administrativo ou norma. Ato eficaz é aquele que está produzindo efeitos.
\newline \textit{Ver também:} \Gls{fim-de-eficacia-expressa}, \Gls{inicio-de-eficacia-expressa}, \Gls{periodo-de-eficacia}, \Gls{vigencia}.},
      }
\newglossaryentry{entrada-em-vigor}{
       name={Entrada em Vigor},
       description={ \textit{Ver:} \Gls{inicio-de-vigencia}.},
      }
\newglossaryentry{excecao-de-aplicacao}{
       name={Exceção de Aplicação},
       description={ \textit{Ver:} \Gls{ressalva-de-aplicacao}.},
      }
\newglossaryentry{exposicao-de-motivos}{
       name={Exposição de Motivos},
       description={Apresentação dos motivos nos quais se baseia um projeto de lei originado fora da Casa Legislativa.
\newline \textit{Ver também:} \Gls{justificativa-da-proposicao-legislativa}, \Gls{proposicao-legislativa}.},
      }
\newglossaryentry{fim-de-eficacia-expressa}{
       name={Fim de Eficácia Expressa},
       description={Evento pelo qual a norma jurídica ou parte dela perde eficácia, mediante disposição expressa.
\newline \textit{Ver também:} \Gls{eficacia}, \Gls{periodo-de-eficacia}.},
      }
\newglossaryentry{fim-de-vacatio-legis}{
       name={Fim de vacatio legis},
       description={Evento que encerra o vacatio legis de norma jurídica ou parte dela. Esse evento é temporalmente contíguo ao Início de Vigência.
\newline \textit{Ver também:} \Gls{inicio-de-vacatio-legis}, \Gls{inicio-de-vigencia}, \Gls{vacatio-legis}.},
      }
\newglossaryentry{fim-de-vigencia}{
       name={Fim de Vigência},
       description={Evento pelo qual a norma jurídica ou parte dela perde a vigência.
\newline \textit{Ver também:} \Gls{periodo-de-vigencia}.},
      }
\newglossaryentry{inicio-de-eficacia-expressa}{
       name={Início de Eficácia Expressa},
       description={Evento pelo qual a norma jurídica ou parte dela passa a ter eficácia, mediante disposição expressa.
\newline \textit{Ver também:} \Gls{eficacia}, \Gls{periodo-de-eficacia}.},
      }
\newglossaryentry{inicio-de-vacatio-legis}{
       name={Início de vacatio legis},
       description={Evento que inicia o período de vacatio legis.
\newline \textit{Ver também:} \Gls{fim-de-vacatio-legis}, \Gls{vacatio-legis}.},
      }
\newglossaryentry{inicio-de-vigencia}{
       name={Início de Vigência},
       description={Evento pelo qual a norma jurídica ou parte dela passa a ter vigência. Esse evento é imediatamente precedido do Fim de vacatio legis, quando houver.
\newline \textit{Ver também:} \Gls{fim-de-vacatio-legis}, \Gls{periodo-de-vigencia}.},
      }
\newglossaryentry{interpretacao-conforme-a-constituicao}{
       name={Interpretação conforme a Constituição},
       description={Técnica de controle de constitucionalidade por meio da qual um tribunal constitucional restringe a aplicação de norma ou dispositivo a determinada interpretação.},
      }
\newglossaryentry{interpretacao-de-ato-normativo}{
       name={Interpretação de ato normativo},
       description={Ato pelo qual se realiza a interpretação de norma jurídica ou dispositivo de forma a uniformizar a sua aplicação.},
      }
\newglossaryentry{jurisprudencia}{
       name={Jurisprudência},
       description={Conjunto das decisões e interpretações das leis feitas pelos tribunais, adaptando as normas às situações de fato.},
      }
\newglossaryentry{justificativa-da-proposicao-legislativa}{
       name={Justificativa da Proposição Legislativa},
       description={Apresentação dos motivos nos quais se baseia um projeto de lei originado no Poder Legislativo.
\newline \textit{Ver também:} \Gls{exposicao-de-motivos}, \Gls{proposicao-legislativa}.},
      }
\newglossaryentry{legislacao-correlata}{
       name={Legislação Correlata},
       description={ \textit{Ver:} \Gls{correlacao}.},
      }
\newglossaryentry{liminar}{
       name={Liminar},
       description={ \textit{Ver:} \Gls{decisao-liminar-em-sede-de-adi}.},
      }
\newglossaryentry{medida-provisoria-prejudicada}{
       name={Medida Provisória Prejudicada},
       description={ \textit{Ver:} \Gls{declaracao-de-prejudicialidade-de-medida-provisoria}.},
      }
\newglossaryentry{mensagem-de-veto}{
       name={Mensagem de Veto},
       description={Comunicação dos motivos que levaram o chefe do Poder Executivo a vetar, total ou parcialmente, um projeto de lei.},
      }
\newglossaryentry{modificacao}{
       name={Modificação},
       description={Considera todas as alterações de texto ou rótulo incluindo os acréscimos, as renumerações e as substituições de expressão.
\newline \textit{Geral de:} \Gls{acrescimo}, \Gls{alteracao}, \Gls{renumeracao}, \Gls{restabelecimento-de-dispositivo}, \Gls{supressao-de-dispositivo}.},
      }
\newglossaryentry{norma-conexa}{
       name={Norma Conexa},
       description={ \textit{Ver:} \Gls{correlacao}.},
      }
\newglossaryentry{norma-juridica}{
       name={Norma Jurídica},
       description={Preceito obrigatório imposto, ou reconhecido como tal, pelo Estado. Para o SILEX, a Norma Jurídica compreende todas as versões da norma no todo ou de seus dispositivos ao longo do tempo, assim como as decisões do tribunal constitucional com efeito vinculante e erga omnes, tais como súmulas vinculantes e decisões em sede de ADI.
\newline \textit{Geral de:} \Gls{acao-direta-de-inconstitucionalidade}, \Gls{acao-declaratoria-de-constitucionalidade}, \Gls{arguicao-de-descumprimento-de-preceito-fundamental}, \Gls{dispositivo-de-norma-juridica}, \Gls{norma-juridica-integral}.
\newline \textit{Ver também:} \Gls{reedicao}, \Gls{versao-de-norma-juridica}.},
      }
\newglossaryentry{norma-juridica-integral}{
       name={Norma Jurídica Integral},
       description={É a Norma Jurídica considerada como um todo, ou seja, o conjunto de todos os seus elementos (epígrafe, ementa, preâmbulo, articulação, fecho, anexos).
\newline \textit{Específico de:} \Gls{norma-juridica}.},
      }
\newglossaryentry{norma-legal}{
       name={Norma Legal},
       description={ \textit{Ver:} \Gls{norma-juridica}.},
      }
\newglossaryentry{nota}{
       name={Nota},
       description={Informação que auxilia ou alerta o entendimento do documento.
\newline \textit{Geral de:} \Gls{nota-de-compilacao}, \Gls{nota-explicativa}.},
      }
\newglossaryentry{nota-de-compilacao}{
       name={Nota de Compilação},
       description={Nota gerada pelo processo de compilação.
\newline \textit{Específico de:} \Gls{nota}.},
      }
\newglossaryentry{nota-explicativa}{
       name={Nota Explicativa},
       description={Nota de texto livre criada pelo mantenedor da informação jurídica.
\newline \textit{Específico de:} \Gls{nota}.},
      }
\newglossaryentry{novo-tratamento-da-materia}{
       name={Novo Tratamento da Matéria},
       description={Caso específico de alerta que descreve uma possível ocorrência de alteração tácita.
\newline \textit{Ver também:} \Gls{alteracao-tacita}, \Gls{revogacao-tacita}.},
      }
\newglossaryentry{ordenamento-juridico}{
       name={Ordenamento Jurídico},
       description={Conjunto de normas jurídicas de um país ou unidade da federação tendo como elemento básico sua Carta Política, e, em harmonia com ela, as leis e demais atos normativos.},
      }
\newglossaryentry{perda-de-eficacia}{
       name={Perda de Eficácia},
       description={ \textit{Ver:} \Gls{fim-de-eficacia-expressa}.},
      }
\newglossaryentry{perempcao-de-concessao-ou-permissao}{
       name={Perempção de Concessão ou Permissão},
       description={Não renovação do prazo de concessão ou permissão pela autoridade competente por falta de conveniência ao interesse nacional, regional ou local, ou ainda pelo não cumprimento por parte do interessado das exigências legais regulamentares aplicáveis ao serviço ou ainda a não observância de suas finalidades educativas ou culturais.
\newline \textit{Referência Legislativa:} Decreto-Lei 227/1967.
\newline \textit{Ver também:} \Gls{caducidade-expressa}, \Gls{declaracao-de-perempcao-de-concessao-ou-permissao}.},
      }
\newglossaryentry{periodico-oficial}{
       name={Periódico Oficial},
       description={Publicação destinada ao conhecimento público dos atos editados sob a responsabilidade e às expensas ou por ordem dos órgãos dos Poderes Executivo, Legislativo e Judiciário, como também de entidades dotadas de personalidade jurídica própria, de qualquer forma vinculadas à administração pública nos níveis federal, estadual e municipal.},
      }
\newglossaryentry{periodo-de-eficacia}{
       name={Período de Eficácia},
       description={Período durante o qual uma norma produz efeitos. No SILEX, o período de eficácia será registrado nos casos expressos.
\newline \textit{Ver também:} \Gls{eficacia}, \Gls{fim-de-eficacia-expressa}, \Gls{inicio-de-eficacia-expressa}, \Gls{periodo-de-vigencia}.},
      }
\newglossaryentry{periodo-de-vigencia}{
       name={Período de Vigência},
       description={Período durante o qual uma norma ou parte dela está em vigor. Em alguns casos, uma norma jurídica pode ter a vigência interrompida, pré determinada ou restabelecida.
\newline \textit{Ver também:} \Gls{fim-de-vigencia}, \Gls{inicio-de-vigencia}, \Gls{periodo-de-eficacia}, \Gls{revigoracao}, \Gls{vigencia}.},
      }
\newglossaryentry{prejudicada}{
       name={Prejudicada},
       description={ \textit{Ver:} \Gls{declaracao-de-prejudicialidade-de-medida-provisoria}.},
      }
\newglossaryentry{promulgacao}{
       name={Promulgação},
       description={Ato pelo qual a autoridade competente dá ciência ao público em geral de que uma norma foi aprovada e entrará em vigor. Atesta a existência da norma.},
      }
\newglossaryentry{promulgacao-de-ato-internacional}{
       name={Promulgação de Ato Internacional},
       description={Ato pelo qual um decreto promulga o texto de um ato internacional, passando a entrar em vigor em território nacional.},
      }
\newglossaryentry{proposicao-legislativa}{
       name={Proposição Legislativa},
       description={Toda matéria sujeita a deliberação de uma casa legislativa que após o devido processo legislativo poderá se transformar em norma jurídica.
\newline \textit{Ver também:} \Gls{dispositivo-de-proposicao-legislativa}, \Gls{exposicao-de-motivos}, \Gls{justificativa-da-proposicao-legislativa}.},
      }
\newglossaryentry{prorrogacao-de-medida-provisoria-por-ato-declaratorio}{
       name={Prorrogação de Medida Provisória},
       description={Evento pelo qual ocorre a prorrogação do prazo de vigência de medida provisória por período determinado. No caso do atual ordenamento jurídico federal, um ato do Presidente do Congresso Nacional dá publicidade a esse evento. (RCN 1/2010 -- art 10).
\newline \textit{Específico de:} \Gls{prorrogacao-de-vigencia-por-prazo-determinado}.},
      }
\newglossaryentry{prorrogacao-de-vigencia}{
       name={Prorrogação de Vigência},
       description={Evento pelo qual a norma ou dispositivo tem o seu período de vigência ampliado.
\newline \textit{Geral de:} \Gls{prorrogacao-de-vigencia-por-prazo-determinado}, \Gls{prorrogacao-de-vigencia-por-prazo-indeterminado}.},
      }
\newglossaryentry{prorrogacao-de-vigencia-por-prazo-determinado}{
       name={Prorrogação de Vigência por Prazo Determinado},
       description={Evento pelo qual a norma ou dispositivo tem o seu período de vigência ampliado por prazo determinado.
\newline \textit{Específico de:} \Gls{prorrogacao-de-vigencia}.
\newline \textit{Geral de:} \Gls{prorrogacao-de-medida-provisoria-por-ato-declaratorio}.},
      }
\newglossaryentry{prorrogacao-de-vigencia-por-prazo-indeterminado}{
       name={Prorrogação de Vigência por Prazo Indeterminado},
       description={Evento pelo qual a norma ou dispositivo tem o seu período de vigência ampliado por prazo indeterminado. Esse relacionamento abrange também aqueles derivados da EMC 32/2001.
\newline \textit{Específico de:} \Gls{prorrogacao-de-vigencia}.},
      }
\newglossaryentry{publicacao}{
       name={Publicação},
       description={Veiculação dos atos normativos em periódico oficial.
\newline \textit{Ver também:} \Gls{publicacao-de-veto-rejeitado}, \Gls{republicacao}, \Gls{republicacao-atualizada}, \Gls{texto-original}.},
      }
\newglossaryentry{publicacao-de-veto-rejeitado}{
       name={Publicação de Veto Rejeitado},
       description={Veiculação, em periódico oficial, de norma ou dispositivo cujo veto foi rejeitado pelo Legislativo.
\newline \textit{Ver também:} \Gls{apreciacao-de-veto}, \Gls{publicacao}, \Gls{republicacao}, \Gls{republicacao-atualizada}, \Gls{veto-parcial}, \Gls{veto-total}.},
      }
\newglossaryentry{publicacao-subsequente}{
       name={Publicação Subsequente},
       description={Veiculações subsequentes do mesmo teor no mesmo periódico oficial ou em periódicos diferentes após uma publicação originária.
\newline \textit{Ver também:} \Gls{republicacao}, \Gls{republicacao-atualizada}, \Gls{texto-original}.},
      }
\newglossaryentry{ratificacao}{
       name={Ratificação},
       description={ \textit{Ver:} \Gls{convalidacao}.},
      }
\newglossaryentry{ratificacao-de-ato-internacional}{
       name={Ratificação de Ato Internacional},
       description={Evento pelo qual o Presidente da República ou, por delegação, o Ministro das Relações Exteriores, confirma a adesão do Brasil ao ato internacional anteriormente assinado. Por ser ato específico da formação de norma internacional, este evento está fora do escopo do SILEX.},
      }
\newglossaryentry{reedicao}{
       name={Reedição},
       description={É a publicação da medida provisória estendendo sua vigência. A reedição de Medida Provisória, com ou sem alteração no seu texto, é considerada como uma Versão de Norma Jurídica Integral e não como uma nova Norma Jurídica.
\newline \textit{Geral de:} \Gls{reedicao-subsequente-com-alteracao}, \Gls{reedicao-subsequente-sem-alteracao}.
\newline \textit{Ver também:} \Gls{norma-juridica}, \Gls{prorrogacao-de-medida-provisoria-por-ato-declaratorio}, \Gls{versao-de-norma-juridica-integral}.},
      }
\newglossaryentry{reedicao-subsequente-com-alteracao}{
       name={Reedição Subsequente com Alteração},
       description={É a publicação da medida provisória estendendo sua vigência com alteração no seu texto em relação à edição imediatamente anterior.
\newline \textit{Específico de:} \Gls{reedicao}.},
      }
\newglossaryentry{reedicao-subsequente-sem-alteracao}{
       name={Reedição Subsequente sem Alteração},
       description={É a publicação da medida provisória estendendo sua vigência sem alteração no seu texto em relação à edição imediatamente anterior.
\newline \textit{Específico de:} \Gls{reedicao}.},
      }
\newglossaryentry{referencia-legislativa}{
       name={Referência Legislativa},
       description={Citação de normas que embasam determinado texto não normativo.},
      }
\newglossaryentry{regime-de-medida-provisoria-ou-decreto-lei}{
       name={Regime de Medida Provisória ou Decreto-Lei},
       description={Conjunto de regras vigente em determinado período que regula a edição, vigência, eficácia, aceitabilidade e conversão das normas com força de lei editadas pelo Chefe do Poder Executivo.},
      }
\newglossaryentry{regulamentacao}{
       name={Regulamentação},
       description={Associação que ocorre entre a norma geral (regulamentada) e a norma específica (regulamentadora) de forma a detalhar a regulamentada para a sua correta execução e/ou aplicação.},
      }
\newglossaryentry{regulamentacao-de-relacoes-juridicas-regidas-por-medida-provisoria-nao-convertida-em-lei}{
       name={Regulamentação de Relações Jurídicas Regidas por MPV não convertida em Lei},
       description={Por força do par 3º do art. 62 da CF, o Congresso Nacional deve editar decreto legislativo para regulamentar as relações jurídicas decorrentes de medida provisória não convertida em lei.},
      }
\newglossaryentry{rejeicao-de-decreto-lei}{
       name={Rejeição de Decreto-Lei},
       description={Ato pelo qual o Poder Legislativo, por meio de um decreto legislativo, rejeita o texto de um decreto-lei.
\newline \textit{Referência Legislativa:} Constituição de 1967, art. 58, § único.},
      }
\newglossaryentry{rejeicao-de-medida-provisoria}{
       name={Rejeição de Medida Provisória},
       description={Ato pelo qual o Poder Legislativo, por meio de um ato declaratório do Presidente da Casa Legislativa, rejeita uma Medida Provisória caso não se verifique os pressupostos de relevância e urgência constitucional. Antes da Resolução CN 1/2002, utilizava-se o termo ``Declaração de Insubsistência de Medida Provisória''.
\newline \textit{Ver também:} \Gls{declaracao-de-insubsistencia-de-medida-provisoria}.},
      }
\newglossaryentry{rejeicao-do-veto}{
       name={Rejeição do Veto},
       description={ \textit{Ver:} \Gls{publicacao-de-veto-rejeitado}.},
      }
\newglossaryentry{remissao}{
       name={Remissão},
       description={Ato ou efeito de encaminhar a um determinado ponto.
\newline \textit{Geral de:} \Gls{remissao-normativa}.
\newline \textit{Ver também:} \Gls{remissao-normativa}.},
      }
\newglossaryentry{remissao-encadeada}{
       name={Remissão Encadeada},
       description={Remissão a dispositivos legislativos que possuem outras remissões (remissão da remissão).
\newline \textit{Específico de:} \Gls{remissao-normativa}.},
      }
\newglossaryentry{remissao-externa}{
       name={Remissão Externa},
       description={Ato ou efeito de encaminhar a um determinado ponto de outro texto normativo.
\newline \textit{Específico de:} \Gls{remissao-normativa}.},
      }
\newglossaryentry{remissao-interna}{
       name={Remissão Interna},
       description={Ato ou efeito de encaminhar a um determinado ponto do próprio texto da norma.
\newline \textit{Específico de:} \Gls{remissao-normativa}.},
      }
\newglossaryentry{remissao-normativa}{
       name={Remissão Normativa},
       description={Ato ou efeito de encaminhar a um determinado ponto de um texto normativo.
\newline \textit{Específico de:} \Gls{remissao}.
\newline \textit{Geral de:} \Gls{remissao-interna}, \Gls{remissao-encadeada}, \Gls{remissao-externa}.},
      }
\newglossaryentry{renumeracao}{
       name={Renumeração},
       description={Alteração no rótulo do dispositivo por comando expresso. A LCP 95/1998 proíbe nos casos de artigos e de unidades superiores a artigos.
\newline \textit{Específico de:} \Gls{modificacao}.},
      }
\newglossaryentry{repristinacao}{
       name={Repristinação},
       description={Revalidação ou volta ao uso ou ao vigor de uma norma jurídica, por revogação ou declaração de inconstitucionalidade da norma revogadora.
\newline \textit{Geral de:} \Gls{repristinacao-expressa}, \Gls{repristinacao-tacita}.},
      }
\newglossaryentry{repristinacao-expressa}{
       name={Repristinação Expressa},
       description={Revalidação ou volta ao uso ou ao vigor de uma norma jurídica, de forma expressa, por revogação ou declaração de inconstitucionalidade da norma revogadora.
\newline \textit{Específico de:} \Gls{repristinacao}.
\newline \textit{Ver também:} \Gls{revigoracao}.},
      }
\newglossaryentry{repristinacao-expressa--com-sentido-de-revigoracao}{
       name={Repristinação Expressa (com sentido de Revigoração)},
       description={ \textit{Ver:} \Gls{revigoracao}.},
      }
\newglossaryentry{repristinacao-tacita}{
       name={Repristinação Tácita},
       description={Revalidação ou volta ao uso ou ao vigor de uma norma jurídica automática. O ordenamento jurídico do Brasil não admite este instituto.
\newline \textit{Específico de:} \Gls{repristinacao}.},
      }
\newglossaryentry{republicacao}{
       name={Republicação},
       description={Publicação do texto de norma jurídica destinada a efetuar correções (LINDB, Art 1º, § 3º).
\newline \textit{Ver também:} \Gls{publicacao}, \Gls{publicacao-de-veto-rejeitado}, \Gls{publicacao-subsequente}, \Gls{republicacao-atualizada}.},
      }
\newglossaryentry{republicacao-atualizada}{
       name={Republicação atualizada},
       description={Ocorre quando a norma determina a republicação nos casos de alterações significativas (Dec. 4.176, art. 25) ou quando a norma é republicada oficialmente com atualização.
\newline \textit{Ver também:} \Gls{publicacao}, \Gls{publicacao-de-veto-rejeitado}, \Gls{publicacao-subsequente}, \Gls{republicacao}.},
      }
\newglossaryentry{reserva-de-aplicacao}{
       name={Reserva de Aplicação},
       description={ \textit{Ver:} \Gls{ressalva-de-aplicacao}.},
      }
\newglossaryentry{ressalva-de-aplicacao}{
       name={Ressalva de Aplicação},
       description={Restrição à aplicação da norma em uma situação expressamente determinada.
\newline \textit{Ver também:} \Gls{alteracao}.},
      }
\newglossaryentry{restabelecimento-de-dispositivo}{
       name={Restabelecimento de Dispositivo},
       description={Caso específico de alteração em que o legislador aproveita um dispositivo revogado estabelecendo nova redação. A alínea c do inciso III do art. 12 da Lei Complementar 95/1998 veda o aproveitamento de dispositivo vetado, revogado, declarado inconstitucional ou de execução suspensa pelo Senado em face de decisão do Supremo Tribunal Federal.
\newline \textit{Específico de:} \Gls{modificacao}.
\newline \textit{Ver também:} \Gls{revigoracao}.},
      }
\newglossaryentry{restabelecimento-de-efeitos}{
       name={Restabelecimento de Efeitos},
       description={Evento pelo qual a norma ou dispositivo anteriormente suspenso volta a produzir efeitos.
\newline \textit{Ver também:} \Gls{revigoracao}, \Gls{suspensao-de-efeitos}.},
      }
\newglossaryentry{restabelecimento-de-eficacia}{
       name={Restabelecimento de Eficácia},
       description={Evento pelo qual a norma ou dispositivo anteriormente suspenso volta a ter eficácia.
\newline \textit{Ver também:} \Gls{suspensao-de-eficacia}.},
      }
\newglossaryentry{restabelecimento-de-vigencia}{
       name={Restabelecimento de Vigência},
       description={ \textit{Ver:} \Gls{revigoracao}.},
      }
\newglossaryentry{restauracao-de-dispositivo}{
       name={Restauração de Dispositivo},
       description={Caso em que o dispositivo volta a ter vigência por conta da rejeição ou decurso de prazo de medida provisória que o alterou ou revogou.},
      }
\newglossaryentry{restricao-de-aplicacao}{
       name={Restrição de aplicação},
       description={ \textit{Ver:} \Gls{ressalva-de-aplicacao}.},
      }
\newglossaryentry{retificacao}{
       name={Retificação},
       description={Ato de corrigir falhas, erros ou omissões nos textos das normas jurídicas anteriormente publicadas.
\newline \textit{Ver também:} \Gls{retificacao--com-sentido-de-alteracao}, \Gls{texto-da-retificacao}.},
      }
\newglossaryentry{retificacao--com-sentido-de-alteracao}{
       name={Retificação (com sentido de Alteração)},
       description={O termo ``retificação'' foi utilizado em normas promulgadas na primeira metade do século XX com o sentido de ``alteração''.
\newline \textit{Ver também:} \Gls{alteracao}, \Gls{retificacao}.},
      }
\newglossaryentry{revigoracao}{
       name={Revigoração},
       description={Evento pelo qual a norma ou dispositivo adquire um novo período de vigência. Excepcionalmente, no âmbito infralegal, utiliza-se a expressão ``restabelecimento de efeitos''.
\newline \textit{Ver também:} \Gls{periodo-de-vigencia}, \Gls{repristinacao-expressa}, \Gls{restabelecimento-de-dispositivo}, \Gls{restabelecimento-de-efeitos}.},
      }
\newglossaryentry{revogacao}{
       name={Revogação},
       description={Evento pelo qual se retira expressamente a vigência de norma no todo ou de dispositivo de norma.
\newline \textit{Geral de:} \Gls{revogacao-com-ressalva}, \Gls{revogacao-condicionada}, \Gls{revogacao-de-norma-integral-com-ressalva}, \Gls{revogacao-parcial}, \Gls{revogacao-tacita}, \Gls{supressao--com-sentido-de-revogacao}.
\newline \textit{Referência Legislativa:} LCP 95/1998, Art. 12, III, c --
            veda o aproveitamento do numero do dispositivo revogado.
\newline \textit{Ver também:} \Gls{anulacao}, \Gls{ato-invalido}, \Gls{supressao--com-sentido-de-revogacao}, \Gls{vigencia}.},
      }
\newglossaryentry{revogacao-com-ressalva}{
       name={Revogação com Ressalva},
       description={Evento em que se retira a vigência da norma, ressalvando determinadas situações.
\newline \textit{Específico de:} \Gls{revogacao}.
\newline \textit{Nota:} Este relacionamento exige a criação de um relacionamento Alerta.
\newline \textit{Ver também:} \Gls{alerta}.},
      }
\newglossaryentry{revogacao-condicionada}{
       name={Revogação Condicionada},
       description={É a revogação que depende da ocorrência de um outro evento. (ex: Lei 5.908/1973 -- art. 9 par 2º a revogação da norma anterior depende da instalação da empresa)
\newline \textit{Específico de:} \Gls{revogacao}.},
      }
\newglossaryentry{revogacao-de-norma-integral-com-ressalva}{
       name={Revogação de Norma Integral com Ressalva},
       description={Evento em que se retira a vigência da norma ressalvando algumas partes, sem a indicação expressa dos dispositivos mantidos (ex: Lei 9472, art. 215, I, revogando CBT com ressalva ``salvo quanto a matéria penal não tratada nesta Lei e quanto aos preceitos relativos à radiodifusão'').
\newline \textit{Específico de:} \Gls{revogacao}.
\newline \textit{Nota:} este relacionamento exige a criação de um relacionamento Alerta.
\newline \textit{Ver também:} \Gls{alerta}.},
      }
\newglossaryentry{revogacao-expressa}{
       name={Revogação Expressa},
       description={ \textit{Ver:} \Gls{revogacao}.},
      }
\newglossaryentry{revogacao-parcial}{
       name={Revogação Parcial},
       description={Relação entre normas derivada dos relacionamentos de Revogação, Revogação Parcial com Ressalva, Revogação da Norma com Ressalva. Trata-se, portanto, de um relacionamento derivado.
\newline \textit{Específico de:} \Gls{revogacao}.
\newline \textit{Ver também:} \Gls{revogacao-total-da-norma}.},
      }
\newglossaryentry{revogacao-por-declaracao}{
       name={Revogação por Declaração},
       description={ \textit{Ver:} \Gls{revogacao}.},
      }
\newglossaryentry{revogacao-por-incompatibilidade}{
       name={Revogação por Incompatibilidade},
       description={ \textit{Ver:} \Gls{revogacao-tacita}.},
      }
\newglossaryentry{revogacao-tacita}{
       name={Revogação Tácita},
       description={Evento em que a norma anterior é revogada quando houver incompatibilidade entre ela e os preceitos da nova norma.
\newline \textit{Específico de:} \Gls{revogacao}.
\newline \textit{Nota:} Esse relacionamento não será registrado no SILEX, pois depende de
            interpretação.
\newline \textit{Ver também:} \Gls{alerta}, \Gls{novo-tratamento-da-materia}.},
      }
\newglossaryentry{revogacao-total-da-norma}{
       name={Revogação Total da Norma},
       description={ \textit{Ver:} \Gls{revogacao}.},
      }
\newglossaryentry{silex}{
       name={SILEX},
       description={Modelo de Requisitos para Sistemas Informatizados de Gestão da Informação Jurídica, elaborado no âmbito do Comitê Gestor de Informação do Portal LexML (CGLEXML)},
      }
\newglossaryentry{situacao-da-norma}{
       name={Situação da Norma},
       description={Exprime o estado ou condição da norma em relação a sua vigência e eficácia.},
      }
\newglossaryentry{supressao--com-sentido-de-revogacao}{
       name={Supressão (com sentido de Revogação)},
       description={O termo ``suprimido'' foi utilizado em normas promulgadas na primeira metade do século XX com o sentido de ``revogação''.
\newline \textit{Específico de:} \Gls{revogacao}.
\newline \textit{Ver também:} \Gls{revogacao}.},
      }
\newglossaryentry{supressao-de-dispositivo}{
       name={Supressão de Dispositivo},
       description={Caso em que o dispositivo acrescido por medida provisória perde a sua vigência por conta da rejeição ou decurso de prazo da medida provisória ou ainda por ausência de sua previsão pela lei de conversão.
\newline \textit{Específico de:} \Gls{modificacao}.},
      }
\newglossaryentry{supressao-de-expressao}{
       name={Supressão de Expressão},
       description={É a alteração expressa que suprime uma expressão no dispositivo.
\newline \textit{Específico de:} \Gls{alteracao-expressa}.},
      }
\newglossaryentry{suspensao-de-efeitos}{
       name={Suspensão de Efeitos},
       description={Relacionamento que suspende os efeitos de norma anterior. Geralmente ocorre no âmbito infralegal com caráter transitório.
\newline \textit{Ver também:} \Gls{restabelecimento-de-efeitos}.},
      }
\newglossaryentry{suspensao-de-eficacia}{
       name={Suspensão de Eficácia},
       description={Evento pelo qual se suspende a aplicação da norma para todos os atos ou fatos por ela regulados. (Vide LCP 95/1998, Art. 12, III, c -- veda o aproveitamento do numero do dispositivo com eficácia suspensa).
\newline \textit{Ver também:} \Gls{decisao-liminar-em-sede-de-adi}, \Gls{restabelecimento-de-eficacia}.},
      }
\newglossaryentry{suspensao-de-execucao}{
       name={Suspensão de Execução},
       description={ \textit{Ver:} \Gls{suspensao-de-eficacia}.},
      }
\newglossaryentry{tecnica-legislativa}{
       name={Técnica Legislativa},
       description={Conjunto de procedimentos e normas redacionais específicas que visam a elaboração de um texto articulado normativo.},
      }
\newglossaryentry{texto}{
       name={Texto},
       description={Expressão linguística composta por uma sequencia de caracteres.
\newline \textit{Geral de:} \Gls{texto--por-derivacao}, \Gls{texto--por-normalizacao}.},
      }
\newglossaryentry{texto--por-derivacao}{
       name={Texto (por Derivação)},
       description={Classificação do conceito Texto na perspectiva da derivação.
\newline \textit{Específico de:} \Gls{texto}.
\newline \textit{Geral de:} \Gls{texto-derivado}, \Gls{texto-original}.},
      }
\newglossaryentry{texto--por-normalizacao}{
       name={Texto (por Normalização)},
       description={Classificação do conceito Texto na perspectiva do processo de normalização.
\newline \textit{Específico de:} \Gls{texto}.
\newline \textit{Geral de:} \Gls{texto-nao-normalizado}, \Gls{texto-normalizado}.},
      }
\newglossaryentry{texto-atualizado}{
       name={Texto Atualizado},
       description={ \textit{Ver:} \Gls{texto-compilado}.},
      }
\newglossaryentry{texto-compilado}{
       name={Texto Compilado},
       description={Texto derivado de uma norma onde se reúnem todas as modificações (alterações, revogações, retificações, etc) para uma determinada data ou para um período.
\newline \textit{Específico de:} \Gls{texto-derivado}.
\newline \textit{Geral de:} \Gls{texto-monovigente}, \Gls{texto-multivigente}, \Gls{texto-retificado}.
\newline \textit{Ver também:} \Gls{alteracao}, \Gls{compilacao}.},
      }
\newglossaryentry{texto-da-republicacao}{
       name={Texto da Republicação},
       description={Texto que veicula a republicação em periódico oficial.
\newline \textit{Específico de:} \Gls{texto-derivado}.},
      }
\newglossaryentry{texto-da-republicacao-com-atualizacao}{
       name={Texto da Republicação com Atualização},
       description={Texto que veicula a versão atualizada de um diploma legal em periódico oficial.
\newline \textit{Específico de:} \Gls{texto-derivado}.},
      }
\newglossaryentry{texto-da-retificacao}{
       name={Texto da Retificação},
       description={Texto que veicula a retificação.
\newline \textit{Ver também:} \Gls{retificacao}, \Gls{texto-retificado}.},
      }
\newglossaryentry{texto-derivado}{
       name={Texto Derivado},
       description={Texto original acrescido das modificações posteriores.
\newline \textit{Específico de:} \Gls{texto--por-derivacao}.
\newline \textit{Geral de:} \Gls{texto-compilado}, \Gls{texto-da-republicacao}, \Gls{texto-da-republicacao-com-atualizacao}.},
      }
\newglossaryentry{texto-monovigente}{
       name={Texto Monovigente},
       description={Texto que traz a redação atualizada da norma para uma determinada data.
\newline \textit{Específico de:} \Gls{texto-compilado}.},
      }
\newglossaryentry{texto-multivigente}{
       name={Texto Multivigente},
       description={Texto que traz a redação atualizada para uma determinada data da norma juntamente com todas as versões dos textos dos dispositivos no período.
\newline \textit{Específico de:} \Gls{texto-compilado}.},
      }
\newglossaryentry{texto-nao-normalizado}{
       name={Texto não Normalizado},
       description={Texto não submetidos ao processo de normalização.
\newline \textit{Específico de:} \Gls{texto--por-normalizacao}.},
      }
\newglossaryentry{texto-normalizado}{
       name={Texto Normalizado},
       description={Texto resultante de um processo de normalização, no qual alguns elementos estruturais são padronizados para a otimização do processamento informatizado. Essa normalização não afeta o conteúdo do dispositivo. Pode ocorrer a padronização de rótulos, o ajuste no rótulo de agrupadores de artigos em relação à capitulação, a marcação de termos em línguas estrangeiras.
\newline \textit{Específico de:} \Gls{texto--por-normalizacao}.},
      }
\newglossaryentry{texto-original}{
       name={Texto Original},
       description={Texto fidedigno à primeira publicação do diploma no periódico oficial sem considerar as retificações e alterações.
\newline \textit{Específico de:} \Gls{texto--por-derivacao}.
\newline \textit{Ver também:} \Gls{publicacao}, \Gls{publicacao-subsequente}, \Gls{texto-retificado}, \Gls{versao-de-norma-juridica}, \Gls{versao-de-norma-juridica-integral}.},
      }
\newglossaryentry{texto-retificado}{
       name={Texto Retificado},
       description={Texto resultante da aplicação da retificação no texto original.
\newline \textit{Específico de:} \Gls{texto-compilado}.
\newline \textit{Ver também:} \Gls{texto-da-retificacao}, \Gls{texto-original}.},
      }
\newglossaryentry{tipo-formal-da-norma}{
       name={Tipo Formal da Norma},
       description={Espécie da norma de acordo com a tipologia documental, tal como consta na epígrafe.
\newline \textit{Ver também:} \Gls{tipo-material-da-norma}.},
      }
\newglossaryentry{tipo-material-da-norma}{
       name={Tipo Material da Norma},
       description={Espécie da norma de acordo com a forma como é incorporada ao ordenamento jurídico. Na maioria das vezes, coincide com o \gls{tipo-formal-da-norma}.
\newline \textit{Ver também:} \Gls{tipo-formal-da-norma}.},
      }
\newglossaryentry{tornar-sem-efeito--com-sentido-de-anulacao}{
       name={Tornar sem efeito (com sentido de Anulação)},
       description={ \textit{Ver:} \Gls{anulacao}.},
      }
\newglossaryentry{tornar-sem-efeito--com-sentido-de-retirada-de-eficacia}{
       name={Tornar sem efeito (com sentido de Retirada de Eficácia)},
       description={Ato pelo qual a autoridade competente retira a eficácia de um ato anterior, como no caso de ato de provimento cuja posse não tenha sido efetivada no prazo previsto.
\newline \textit{Referência Legislativa:} Lei 8112/90, art. 13, § 6º.},
      }
\newglossaryentry{tornar-sem-efeito--com-sentido-de-revogacao}{
       name={Tornar sem efeito (com sentido de Revogação)},
       description={ \textit{Ver:} \Gls{revogacao}.},
      }
\newglossaryentry{unidade-de-informacao}{
       name={Unidade de Informação},
       description={Documento ou parte específica do documento que possui um nome único.},
      }
\newglossaryentry{vacancia}{
       name={Vacância},
       description={ \textit{Ver:} \Gls{vacatio-legis}.},
      }
\newglossaryentry{vacatio-legis}{
       name={vacatio legis},
       description={Período entre a publicação e a entrada em vigor da norma.},
      }
\newglossaryentry{versao-de-dispositivo-de-norma-juridica}{
       name={Versão de Dispositivo de Norma Jurídica},
       description={Versão do comando legal veiculado por um texto vigente de um dispositivo em uma determinada data.
\newline \textit{Específico de:} \Gls{versao-de-norma-juridica}.
\newline \textit{Ver também:} \Gls{dispositivo-de-norma-juridica}.},
      }
\newglossaryentry{versao-de-norma-juridica}{
       name={Versão de Norma Jurídica},
       description={Versão do comando legal veiculado por um texto vigente em uma determinada data.
\newline \textit{Geral de:} \Gls{versao-de-norma-juridica-integral}, \Gls{versao-de-dispositivo-de-norma-juridica}.
\newline \textit{Ver também:} \Gls{norma-juridica}, \Gls{texto-derivado}, \Gls{texto-original}.},
      }
\newglossaryentry{versao-de-norma-juridica-integral}{
       name={Versão de Norma Jurídica Integral},
       description={Versão de uma Norma Jurídica considerada como um todo em uma determinada data.
\newline \textit{Específico de:} \Gls{versao-de-norma-juridica}.
\newline \textit{Ver também:} \Gls{norma-juridica-integral}, \Gls{reedicao}, \Gls{texto-derivado}, \Gls{texto-original}.},
      }
\newglossaryentry{veto}{
       name={Veto},
       description={Ato pelo qual o Chefe do Poder Executivo nega sanção ao Projeto -- ou a parte dele --, obstando à sua conversão em norma jurídica. (Manual Presidência...) (Vide CF, art. 66, Par 1º) (Vide LCP 95/1998, Art. 12, III, c -- veda o aproveitamento do numero do dispositivo vetado).
\newline \textit{Geral de:} \Gls{veto-parcial}, \Gls{veto-total}.},
      }
\newglossaryentry{veto-de-expressao}{
       name={Veto de Expressão},
       description={É o veto parcial que abrange expressões ou palavras dos dispositivos. A partir do início da vigência da Emenda 17/1965 da CF de 1946, proibiu-se o veto de expressão.
\newline \textit{Específico de:} \Gls{veto-parcial}.},
      }
\newglossaryentry{veto-parcial}{
       name={Veto Parcial},
       description={É o veto que abrange o texto integral de artigo, de parágrafo, de inciso ou de alínea, podendo abranger, antes da vigência da Emenda 17/1965 da CF de 1946, expressões ou palavra dos dispositivos. Neste último caso, deve-se utilizar o relacionamento ``Veto de Expressão''. (Vide CF, art. 66, Par 2º).
\newline \textit{Específico de:} \Gls{veto}.
\newline \textit{Geral de:} \Gls{veto-de-expressao}.
\newline \textit{Ver também:} \Gls{publicacao-de-veto-rejeitado}, \Gls{rejeicao-do-veto}.},
      }
\newglossaryentry{veto-total}{
       name={Veto Total},
       description={É o veto que abrange todo o texto da proposição legislativa.
\newline \textit{Específico de:} \Gls{veto}.
\newline \textit{Ver também:} \Gls{publicacao-de-veto-rejeitado}, \Gls{rejeicao-do-veto}.},
      }
\newglossaryentry{vigencia}{
       name={Vigência},
       description={É a existência da norma no ordenamento jurídico por um ou mais períodos temporais. Uma norma vige até que outra a revogue ou até que expire o prazo nela previsto. Situa-se a vigência como marco intermédio entre a existência, que se formaliza pela promulgação, e a eficácia, que decorre da observância social da norma. (Enc. Saraiva, v. 77).
\newline \textit{Ver também:} \Gls{eficacia}, \Gls{periodo-de-vigencia}, \Gls{revogacao}.},
      }
