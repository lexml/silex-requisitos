% ----------------------------------------------------------
\chapter{Tabelas de relacionamentos}
\label{apendices_tabelas_relacionamentos}
% ----------------------------------------------------------

\newcommand{\mypar}{\par}

% --
\begin{tabelarelacionamento}{Relacionamentos
de Publicação}{tab-relacionamento-publicacao}

 \relacionamento
 	{Versão da Norma Jurídica}
 	{PUBLICADA EM (1.. 1)}
 	{Publicação}
 	{(1.. n) Publicação (1..n) 
 	 \mypar 
 	 [ data, edição, \{ pagina, coluna \} inicial e final]}
 	{PUBLICAÇÃO DE (1..1)}
 	{Documento da Publicação Oficial (seção, edição extra, suplemento)}
 	{D: ( Publicado no <doc>, em <data>, na p. <pagInicial>[-<pagFinal>)}
 	
 \relacionamento
 	{Versão da Norma Jurídica}
 	{PUBLICADA NOVAMENTE EM (1.. 1)}
 	{Publicação Subsequente}
 	{(1.. n) Publicação Subsequente (1..n) 
 	 \mypar 
 	 [ data, edição, \{ pagina, coluna \} inicial e final ]}
 	{PUBLICAÇÃO SUBSEQUENTE DE (1..1)}
 	{Documento da Publicação Oficial (seção, edição extra, suplemento)}
 	{D: ( Publicado novamente no <doc>, em <data>, na p. <pagInicial>[-<pagFinal>)}
 	
 \relacionamento
 	{Versão da Norma Jurídica}
 	{PUBLICADA EM (1.. 1)}
 	{Publicação de Veto Rejeitado}
 	{(1.. n) Publicação de Veto Rejeitado (1..n)
 	 \mypar 
 	 [ data, edição, \{ pagina, coluna \} inicial e final, dataIniVigencia ]}
 	{PUBLICAÇÃO DE (1..1)}
 	{Documento da Publicação Oficial (seção, edição extra, suplemento)}
 	{D: ( Publicado no <doc>, em <data>, na p. <pagInicial>[-<pagFinal> com vigência a partir de <dataVig>)}
 	
 \relacionamento
 	{Versão da Norma Jurídica}
 	{REPUBLICADA EM (1..1)}
 	{Republicação}
 	{(1..n) Republicação (1..n)
 	 \mypar 
 	 [ data, edição, \{pagina, coluna\} inicial e final ]}
 	{REPUBLICAÇÃO DE (1..1)}
 	{Documento da Publicação Oficial (seção, edição extra, suplemento)}
 	{D: ( Republicado no <doc>, em <data>, na p. <pagInicial>[-<pagFinal>])}
 	
 \relacionamento
 	{Versão da Norma Jurídica}
 	{REPUBLICADA COM ATUALIZAÇÃO EM (1..1)}
 	{Republicação Atualizada}
 	{(1..n) Republicação Atualizada (1..n)
 	 \mypar 
 	 [ data, edição, \{pagina, coluna\} inicial e final ]}
 	{REPUBLICAÇÃO COM ATUALIZAÇÃO DE (1..1)}
 	{Documento da Publicação Oficial (seção, edição extra, suplemento)}
 	{D: ( Republicado com atualização no <doc>, em <data>, na p. <pagInicial>[-<pagFinal>])}
 	
 \relacionamento
 	{Versão da Norma Jurídica}
 	{RETIFICADO (1..1)}
 	{Retificação}
 	{(1..n) Retificação (1..n)
 	 \mypar 
 	 [ data, edição, \{pagina, coluna\} inicial e final ]}
 	{RETIFICADOR (1..1)}
 	{Documento da Publicação Oficial (seção, edição extra, suplemento)}
 	{D: ( Retificado no <doc>, em <data>, na p. <pagInicial>[-<pagFinal>])}

\end{tabelarelacionamento}

% --
\begin{tabelarelacionamento}{Relacionamentos
de  Modificação}{tab-relacionamento-modificacao}

 \relacionamento
 	{Dispositivo de norma jurídica}
 	{ALTERADO (1..1)}
 	{Alteração Expressa de Dispositivo}
 	{(1..n) Alteração Expressa de Dispositivo (1..n) 
 	 \mypar 
 	 [ Tipo: NR, AC + NR ( monovigente) ]}
 	{Alterador (1..1)}
 	{Versão de Dispositivo de Norma Jurídica}
 	{D: (Redação dada pela <ref>)
 	 \mypar
 	 D: (Acrescido pela <refSemVigencia> e alterado pela <ref>)}
 	  	 
 \relacionamento
 	{Dispositivo de norma jurídica}
 	{ALTERADO (1..1)}
 	{Alteração Expressa de Expressão}
 	{(1..n) Alteração Expressa de Expressão (1..n) 
 	 \mypar 
 	 [ Tipo: NR ]}
 	{Alterador (1..1)}
 	{Versão de Dispositivo de norma jurídica}
 	{D: (Expressão <exp> alterada <ref>)}

 \relacionamento
 	{Dispositivo de norma jurídica}
 	{ALTERADO (1..1)}
 	{Supressão de Expressão}
 	{(1..n) Supressão de Expressão (1..n) 
 	 \mypar 
 	 [ Tipo: NR ]}
 	{Alterador (1..1)}
 	{Versão de Dispositivo de norma jurídica}
 	{D: (Expressão <exp> suprimida pela <ref>)}

 \relacionamento
 	{Dispositivo de norma jurídica}
 	{ACRESCIDO (1..1)}
 	{Acréscimo de Dispositivo}
 	{(1..n) Acréscimo de Dispositivo (1..n) 
 	 \mypar 
 	 [ Tipo: AC ]}
 	{ACRESCENTADOR (1..1)}
 	{Versão de Dispositivo de norma jurídica}
 	{D: (Acrescido pela <ref>)}

 \relacionamento
 	{Dispositivo de norma jurídica}
 	{ACRESCIDO (1..1)}
 	{Acréscimo de Expressão}
 	{(1..n) Acréscimo de Expressão (1..n) 
 	 \mypar 
 	 [ Tipo: AC ]}
 	{ACRESCENTADOR (1..1)}
 	{Versão de Dispositivo de norma jurídica}
 	{D: (Expressão <exp> acrescida pela <ref>)}

 \relacionamento
 	{Dispositivo de norma jurídica}
 	{RENUMERADO (1..1)}
 	{Renumeração}
 	{(1..n) Renumeração (1..n)}
 	{RENUMERADOR (1..1)}
 	{Versão de Dispositivo de norma jurídica}
 	{D: (Renumerado por <ref>)}

 \relacionamento
 	{Dispositivo de norma jurídica}
 	{RESTABELECIDO (1..1)}
 	{Restabelecimento de Dispositivo}
 	{(1..n) Restabelecimento de Dispositivo (1..n)}
 	{RESTABELECEDOR (1..1)}
 	{Versão de Dispositivo de norma jurídica}
 	{D: (Restabelecido por <ref>)}

 \relacionamento
 	{Dispositivo de norma jurídica}
 	{SUPRIMIDO (1..1)}
 	{Supressão de Dispositivo}
 	{(1..n) Supressão de Dispositivo (1..n)}
 	{SUPRESSOR (1..1)}
 	{Versão de Dispositivo de norma jurídica}
 	{D: (Suprimido por <refAtoRejeicao>, <refLeiConversao> ou <refAtoDeclaratorio>)}
 	
\end{tabelarelacionamento}

% --
\begin{tabelarelacionamento}{Relacionamentos
de Vigência}{tab-relacionamento-vigencia}

  \relacionamento
  	{Norma jurídica}
 	{PEREMPTA (1..1)}
 	{Declaração de Perempção de Concessão ou Permissão}
 	{(1..n) Declaração de Perempção de Concessão ou Permissão (1..n)}
 	{DECLARA PEREMPTA (1..1)}
 	{Versão de Dispositivo de norma jurídica}
 	{D: (Declarada perempta em dd.mm.aaaa, de acordo com o <refDisp> )}

  \relacionamento
  	{Norma jurídica}
 	{VIGÊNCIA PRORROGADA POR PRAZO DETERMINADO (1..1)}
 	{Prorrogação de Vigência por prazo determinado}
 	{(1..n) Prorrogação de Vigência por prazo determinado (1.. n)
 	 \mypar
 	 [ DataInicio, DataFim ]}
 	{PRORROGA VIGÊNCIA POR PRAZO DETERMINADO  (1..1)}
 	{Versão de Dispositivo de norma jurídica}
 	{D: (Vigência prorrogada de \{dataInicio\} até \{dataFim\}. Ver: <ref>)}

  \relacionamento
  	{Norma jurídica}
 	{VIGÊNCIA PRORROGADA POR PRAZO INDETERMINADO (1..1)}
 	{Prorrogação de Vigência por prazo indeterminado}
 	{(1..n) Prorrogação de Vigência por prazo indeterminado (1.. n)
 	 \mypar
 	 [ DataInicio ]}
 	{PRORROGA VIGÊNCIA POR PRAZO INDETERMINADO (1..1)}
 	{Versão de Dispositivo de norma jurídica}
 	{D: (Vigência prorrogada a partir de \{dataInicio\}. Ver: <ref>)}

  \relacionamento
  	{Norma jurídica}
 	{REPRISTINADO (1..1)}
 	{Repristinação Expressa}
 	{(1..1) Repristinação Expressa (1..n)
 	 \mypar
 	 \{ <trib>, <ref>, <pub> \}}
 	{REPRISTINADOR (1..1)}
 	{Acórdão em Controle Concentrado de Constitucionalidade; Norma Jurídica}
 	{D: (Repristinação de dispositivo por Declaração de Inconstitucionalidade pelo <trib>, <ref>,  publicado no <pub>)}

  \relacionamento
  	{Dispositivo de norma jurídica}
 	{RESTAURADO (1..1)}
 	{Restauração de Dispositivo}
 	{(1..n) Restauração de Dispositivo (1..n)}
 	{RESTAURADOR (1..1)}
 	{Versão de Dispositivo de norma jurídica }
 	{D: (Restaurado por <refAtoRejeicao>, <refLeiConversao> ou <refAtoDeclaratorio>)}

  \relacionamento
  	{Norma Jurídica}
 	{REVIGORADO (1..1)}
 	{Revigoração}
 	{(1..n) Revigoração (1..n)}
 	{REVIGORADOR (1..1)}
 	{Versão de Dispositivo de Norma Jurídica}
 	{D: (Revigorado por <ref>)}

  \relacionamento
  	{Norma Jurídica}
 	{REVOGADO (1..1)}
 	{Revogação}
 	{(1..n) Revogação (1..n)}
 	{REVOGADOR (1..1)}
 	{Versão de Dispositivo de Norma Jurídica}
 	{D: (Revogado por <ref>)}

  \relacionamento
  	{Norma Jurídica}
 	{REVOGADO (1..1)}
 	{Revogação Condicionada}
 	{(1..n) Revogação Condicionada (1..n)
 	 \mypar
 	 \{ evento determinante \}}
 	{REVOGADOR (1..1)}
 	{Versão de Dispositivo de Norma Jurídica}
 	{D: (Revogado por <ref>)}

  \relacionamento
  	{Norma Jurídica}
 	{REVOGADO (1..1)}
 	{Revogação Parcial}
 	{(1..n) Revogação Parcial (1..n)
 	 \mypar
 	 [ RELACIONAMENTO DERIVADO de Revogação, Revogação com Ressalva, Revogação da Norma Integral com ressalva ]}
 	{REVOGADOR (1..1)}
 	{Norma Jurídica}
 	{D: (Revogado por <ref>)}

  \relacionamento
  	{Norma Jurídica}
 	{REVOGADO (1..1)}
 	{Revogação com Ressalva}
 	{(1..n) Revogação com Ressalva (1..n)}
 	{REVOGADOR (1..1)}
 	{Versão de Dispositivo de Norma Jurídica}
 	{D: (Revogado com ressalva por <ref>)}

  \relacionamento
  	{Norma Jurídica}
 	{REVOGADO (1..1)}
 	{Revogação da Norma Integral com Ressalva}
 	{(1..n) Revogação da Norma Integral com Ressalva (1..n)}
 	{REVOGADOR (1..1)}
 	{Versão de Dispositivo de Norma Jurídica}
 	{D: (Revogado com ressalva por <ref>}

\end{tabelarelacionamento}


% --
\begin{tabelarelacionamento}{Relacionamentos
de Eficácia}{tab-relacionamento-eficacia}

  \relacionamento
  	{Norma jurídica}
 	{CADUCA (1..1)}
 	{Declaração de Caducidade}
 	{(1..n) Declaração de Caducidade (1..n)}
 	{DECLARA CADUCO  (1..1)}
 	{Versão de Dispositivo de norma jurídica}
 	{D: (Declarada caduca em dd.mm.aaaa, de acordo com o <refDisp> )}

  \relacionamento
  	{Norma jurídica}
 	{EFICAZ (1..1)}
 	{Declaração de Eficácia Expressa}
 	{(1..n) Declaração de Eficácia Expressa (1..n)}
 	{DECLARA EFICAZ  (1..1)}
 	{Versão de Dispositivo de norma jurídica}
 	{D: (Declaro eficaz a partir de dd.mm.aaaa, de acordo com o <refDisp> )}
 	
  \relacionamento
  	{Norma jurídica}
 	{SUSPENSO (1..1)}
 	{Suspensão de Efeitos}
 	{(1..n) Suspensão de Efeitos (1..n)}
 	{NORMA SUSPENSORA (1..1)}
 	{Versão de Dispositivo de norma jurídica}
 	{(Suspensão de efeitos pela <ref>)}
 	
  \relacionamento
  	{Norma jurídica}
 	{RESTABELECIDO (1..1)}
 	{Restabelecimento de Efeitos}
 	{(1..n) Restabelecimento de Efeitos (1..n)}
 	{RESTABELECEDOR (1..1)}
 	{Versão de Dispositivo de norma jurídica}
 	{(Restabelecimento de efeitos pela <ref>)}
 	
  \relacionamento
  	{Norma jurídica}
 	{SUSPENSA (1..1)}
 	{Suspensão de Eficácia}
 	{(1..n) Suspensão de Eficácia (1..1)
	 \mypar
	 (<orgao>) <disp><ref>, <pub>}
 	{NORMA SUSPENSORA (1..1)}
 	{Resolução do SF; Resolução de Assembleia Legislativa; Resolução de Câmara Distrital; Liminar em Sede de ADIN}
 	{D: (Execução suspensa por  órgão na forma do <disp>, <ref>, <pub>)}
 	
  \relacionamento
  	{Norma jurídica}
 	{RESTABELECIDO (1..1)}
 	{Restabelecimento de Eficácia}
 	{(1..n) Restabelecimento de Eficácia (1..1)}
 	{RESTABELECEDOR (1..1)}
 	{Resolução do SF; Resolução de Assembleia Legislativa; Resolução de Câmara Distrital; Decisão em Sede de ADIN}
 	{D: (Eficácia restabelecida pela  <ref>)}
 	
  \relacionamento
  	{Norma jurídica}
 	{SEM EFEITO (1..1)}
 	{Tornar sem efeito (com sentido de retirada de eficácia)}
 	{(1..n) Tornar sem efeito (com sentido de retirada de eficácia) (1..1)}
 	{RETIRADA DE EFEITO (1..1)}
 	{Versão de Dispositivo de Norma Jurídica}
 	{D: (Eficácia restabelecida pela  <ref>)}
 	
\end{tabelarelacionamento}

% --
\begin{tabelarelacionamento}{Relacionamentos
de Períodos de Vacatio, Vigência e Eficácia}{tab-relacionamento-vacatio}

  \relacionamento
  	{Norma jurídica}
 	{PERDEU A EFICÁCIA (1..1)}
 	{Fim de Eficácia Expressa}
 	{(1..n) Fim de Eficácia Expressa (1..n)
 	 \mypar
 	 \{decurso de prazo\}}
 	{RETIROU A EFICÁCIA (1..1)}
 	{Versão de Dispositivo de norma jurídica}
 	{D: (Fim de Eficácia em dd.mm.aaaa, de acordo com o <refDisp> )}

  \relacionamento
  	{Norma jurídica}
 	{ENCERRA A VACATIO (1..1)}
 	{Fim de Vacatio Legis}
 	{(1..n) Fim de Vacatio Legis (1..n)
 	 \mypar
 	 \{decurso de prazo\}}
 	{DETERMINA VACATIO (1..1)}
 	{Versão de Dispositivo de norma jurídica}
 	{D: (Fim de vacatio legis em dd.mm.aaaa, de acordo com o <refDisp> )}


  \relacionamento
  	{Norma jurídica}
 	{PERDEU A VIGÊNCIA (1..1)}
 	{Fim de Vigência}
 	{(1..n) Fim de Vigência (1..n)}
 	{RETIROU A VIGÊNCIA (1..1)}
 	{Versão de Dispositivo de norma jurídica}
 	{D: (Fim de Vigência em dd.mm.aaaa, de acordo com o <refDisp> )}

  \relacionamento
  	{Norma jurídica}
 	{INICIA A EFICÁCIA (1..1)}
 	{Início de Eficácia Expressa}
 	{(1..n) Início de Eficácia Expressa (1..n)}
 	{ESTABELECE EFICÁCIA (1..1)}
 	{Versão de Dispositivo de norma jurídica}
 	{D: (Início de Eficácia em dd.mm.aaaa, de acordo com o <refDisp> )}

  \relacionamento
  	{Norma jurídica}
 	{INICIA A VACATIO (1..1)}	
 	{Início de Vacatio Legis}
 	{(1..n) Início de Vacatio Legis (1..n)}
 	{DETERMINA VACATIO (1..1)}
 	{Versão de Dispositivo de norma jurídica}
 	{D: (Início de vacatio legis em dd.mm.aaaa, de acordo com o <refDisp> )}

  \relacionamento
  	{Norma jurídica}
 	{INICIA A VIGÊNCIA (1..1)}
 	{Início de Vigência}
 	{(1..n) Início de Vigência (1..n)}
 	{ESTABELECE VIGÊNCIA (1..1)}
 	{Versão de Dispositivo de norma jurídica}
 	{D: (Início de Vigência em dd.mm.aaaa, de acordo com o <refDisp> )}
\end{tabelarelacionamento}


% --
\begin{tabelarelacionamento}{Relacionamentos
de Veto}{tab-relacionamento-veto}

  \relacionamento
  	{Parte de Dispositivo de Proposição Legislativa}
 	{VETADO (1..1)}
 	{Veto de Expressão}
 	{(1..1) Veto de Expressão (1..n)}
 	{ATO DO VETO}
 	{Mensagem do Poder Executivo}
 	{D: (Vetado por <ref>)}

  \relacionamento
  	{Dispositivo de Proposição Legislativa}
 	{VETADO (1..1)}
 	{Veto Parcial}
 	{(1..1) Veto Parcial (1..n)}
 	{ATO DO VETO}
 	{Mensagem do Poder Executivo}
 	{D: (Vetado por <ref>)}
 	
  \relacionamento
  	{Proposição Legislativa}
 	{VETADA (1..1)}
 	{Veto Total}
 	{(1..1) Veto Total (1..1)}
 	{ATO DO VETO}
 	{Mensagem do Poder Executivo}
 	{D: (Vetado por <ref>)}
 	
\end{tabelarelacionamento} 	


% --
\begin{tabelarelacionamento}{Relacionamentos
de Anotações}{tab-relacionamento-anotacoes}

  \relacionamento
  	{Entidade}
 	{DESTINATÁRIA (1..1)}
 	{Alerta}
 	{(1..n) Alerta ( 1..n)}
 	{ORIGINÁRIA (1..1)}
 	{Entidade}
 	{D: (Alerta:  <texto >)}

  \relacionamento
  	{Entidade}
 	{CORRELATA A (1..1)}
 	{Correlação}
 	{(1..n) Correlação (1..n)
	 \mypar
	 \{ simétrica , não transitiva\}}
 	{CORRELATA A (1..1)}
 	{Entidade}
 	{D/I: ( Vide: <ref> )}

  \relacionamento
  	{Norma Jurídica}
 	{POSSUI NOVO TRATAMENTO (1..1)}
 	{Novo Tratamento da Matéria}
 	{(1..n) Novo Tratamento da Matéria (1..n)}
 	{DÁ NOVO TRATAMENTO (1..1)}
 	{Versão de Norma Jurídica}
 	{D: (Novo tratamento da matéria por <ref> )
	 \mypar
	 I: (Ver também: <ref>)}

  \relacionamento
  	{Norma Jurídica}
 	{REGULAMENTADO (1..1)}
 	{Regulamentação}
 	{(1..n) Regulamentação (1..n)}
 	{REGULAMENTADOR (1..1)}
 	{Versão de Norma Jurídica}
 	{D: (Regulamentado por <ref>)
	 \mypar
	 I: (Regulamenta <ref> )}

  \relacionamento
  	{Norma Jurídica}
 	{RESSALVADA  (1..1)}
 	{Ressalva de Aplicação}
 	{(1..n) Ressalva de Aplicação  (1..n)}
 	{RESSALVA (1..1)}
 	{Versão de Dispositivo de Norma Jurídica}
 	{D: ( Ressalva de Aplicação conforme <ref> )}
\end{tabelarelacionamento}


% --
\begin{tabelarelacionamento}{Relacionamentos
de Validade/Nulidade}{tab-relacionamento-validade}

  \relacionamento
  	{Ato Administrativo}
 	{ANULADO (1..1)}
 	{Anulação}
 	{(1..1) Anulação (1..n)}
 	{ANULADOR (1..1)}
 	{Versão de Norma Jurídica; Acórdão; Ato Administrativo}
 	{D: (Anulado por <ref> )}
 	
  \relacionamento
  	{Entidade com Vigência}
 	{CONVALIDADA (1..1)}
 	{Convalidação}
 	{(1..1) Convalidação (1..n)}
 	{CONVALIDANTE (1..1)}
 	{Entidade com Vigência}
 	{D: (Convalidado por <ref> )}
 	
  \relacionamento
  	{Norma Jurídica; Parte do Dispositivo de Norma Jurídica}
 	{DECLARADA INCONSTITUCIONAL (1..1)}
 	{Declaração de Inconstitucionalidade}
 	{(1..1) Declaração de Inconstitucionalidade (1..n)
	 \mypar
	 \{ <trib>, <ref>, <pub> \}}
 	{DECLARA INCONSTITUCIONAL (1..1)}
 	{Acórdão em Controle Concentrado de Constitucionalidade}
 	{D: (Declarado Inconstitucional pelo <trib>, <ref>,  publicado no <pub>)}
 	
  \relacionamento
  	{Norma Jurídica}
 	{DECLARADA NÃO RECEPCIONADA (1..1)}
 	{Declaração de não recepção}
 	{(1..1) Declaração de não recepção (1..n)
	 \mypar
	 \{ <esfera> <ano> <trib>, <ref>, <pub> \} }
 	{DECLARA NÃO RECEPCIONADA (1..1)}
 	{Acórdão em Controle Concentrado de Constitucionalidade}
 	{D: (Declarado não recepcionado pela Constituição <esfera> <ano> em controle concentrado pelo <trib>, <ref>, publicado no <pub>)}
 	
\end{tabelarelacionamento}

% --
\begin{tabelarelacionamento}{Relacionamentos
de Interpretação}{tab-relacionamento-interpretacao}

  \relacionamento
  	{Norma Jurídica}
 	{INTERPRETADO (1..1)}
 	{Interpretação de ato normativo}
 	{(1..n) Interpretação de ato normativo (1..n)}
 	{INTERPRETADOR (1..1)}
 	{Versão da Norma Jurídica}
 	{D: (Interpretação conforme o <ref> )}

  \relacionamento
  	{Norma Jurídica}
 	{INTERPRETADO CONFORME A CONSTITUIÇÃO (1..1)}
 	{Interpretação conforme a Constituição}
 	{(1..n) Interpretação conforme a Constituição (1..n)}
 	{INTERPRETADOR (1..1)}
 	{Acórdão}
 	{D: (Interpretação conforme a Constituição pelo <ref>. [ Ementa da Decisão ] )}

\end{tabelarelacionamento}


% --
\begin{tabelarelacionamento}{Relacionamentos
de Medidas Provisórias/Decreto-Lei}{tab-relacionamento-medidas-provisorias}

  \relacionamento
  	{Medida Provisória (como Versão de Norma Jurídica)}
 	{CONVERTIDA COM ALTERAÇÃO (1..1)}
 	{Conversão com Alteração}
 	{(1..1) Conversão com Alteração (1..1)}
 	{ORIGINÁRIA DA (1..1)}
 	{Lei (como Versão de Norma Jurídica)}
 	{D: (Convertida com alteração na <ref> )
	 \mypar
	 I: (Conversão com alteração da <ref>)}

  \relacionamento
  	{Decreto-Lei (como Versão de Norma Jurídica)}
 	{CONVERTIDO (1..1)}
 	{Conversão de Decreto-Lei em Medida-Provisória}
 	{(1..1) Conversão de Decreto-Lei em Medida-Provisória (1..1)}
 	{ORIGINÁRIA DO (1..1)}
 	{Medida Provisória (como Versão de Norma Jurídica)}
 	{D: (Convertido na <ref> )
	 \mypar
	 I: (Originária do <ref>)}

  \relacionamento
  	{Medida Provisória (como Versão de Norma Jurídica)}
 	{CONVERTIDA SEM ALTERAÇÃO (1..1)}
 	{Conversão sem Alteração}
 	{(1..1) Conversão sem Alteração (1..1)}
 	{ORIGINÁRIA DA (1..1)}
 	{Lei  (como Versão de Norma Jurídica)}
 	{D: (Convertida sem alteração na <ref> )
	 \mypar
	 I: (Conversão sem alteração da <ref>)}

  \relacionamento
  	{Medida Provisória (como Norma Jurídica)}
 	{DECLARADA INSUBSISTENTE (1..1)}
 	{Declaração de Insubsistência de MP}
 	{(1..1) Declaração de Insubsistência de MP (1..n)}
 	{DECLARA INSUBSISTENTE (1..1)}
 	{Resolução do CN (como Norma Jurídica)}
 	{D: (Declarado Insubsistente pela <ref> )}

  \relacionamento
  	{Medida Provisória (como Norma Jurídica)}
 	{DECLARADA PREJUDICADA (1..1)}
 	{Declaração de Prejudicialidade de Medida Provisória}
 	{(1..1) Declaração de Prejudicialidade de Medida Provisória (1..n)}
 	{DECLARA PREJUDICADA (1..1)}
 	{Ato (como Norma Jurídica)}
 	{D: (Declarada Prejudicada pela <ref> )}

  \relacionamento
  	{Medida Provisória (como Norma Jurídica)}
 	{DISCIPLINADA (1..1)}
    {Disciplinamento de Relações Jurídicas Decorrentes de Medidas Provisórias Rejeitadas ou Declaradas Insubsistentes} 
    {(1..1) Disciplinamento de Relações Jurídicas Decorrentes de Medidas Provisórias Rejeitadas ou Declaradas Insubsistentes (1..n)}
 	{DISCIPLINA (1..1)}
 	{Decreto Legislativo (como Norma Jurídica)}
 	{D: (Relações jurídicas disciplinadas  por <ref>)
	 \mypar
	 I: (Disciplina as relações jurídicas de <ref>)}

  \relacionamento
  	{Medida Provisória (como Versão de Norma Jurídica)}
 	{REEDITADA SEM ALTERAÇÃO  (1..1)}
 	{Reedição Subseqüente sem Alteração}
 	{(1..1) Reedição Subseqüente sem Alteração (1..1)}
 	{REEDIÇÃO SEM ALTERAÇÃO  DE  (1..1)}
 	{Medida Provisória (como Versão de Norma Jurídica)}
 	{D: (Reeditada sem alteração: <ref> )
	 \mypar
	 I: (Reedição sem alteração da <ref>)}

  \relacionamento
  	{Medida Provisória (como Versão de Norma Jurídica)}
 	{REEDITADA COM ALTERAÇÃO (1..1)}
 	{Reedição Subseqüente com Alteração}
 	{(1..1) Reedição Subseqüente com Alteração (1..1)}
 	{REEDIÇÃO COM ALTERAÇÃO DE  (1..1)}
 	{Medida Provisória (como Versão de Norma Jurídica)}
 	{D: (Reeditada com alteração: <ref> )
	 \mypar
	 I: (Reedição com alteração da <ref>)}

  \relacionamento
  	{Medida Provisória (como Versão de Norma Jurídica)}
 	{RELACIONADA A (1..1)}
 	{Regulamentação de Relações Jurídicas Regidas por Medida Provisória não Convertida em Lei}
 	{Regulamentação de Relações Jurídicas Regidas por Medida Provisória não Convertida em Lei}
 	{EDIÇÃO DE (1..1)}
 	{Decreto Legislativo (como Versão de Norma Jurídica)}
 	{D: (Relações jurídicas regulamentadas por <ref>)
	 \mypar
	 I: (Regulamenta as relações jurídicas de <ref>)}

  \relacionamento
  	{Decreto-Lei (como Norma Jurídica)}
 	{REJEITADO (1..1)}
 	{Rejeição de Decreto-Lei}
 	{(1..1) Rejeição de Decreto-Lei (1..1)}
 	{REJEITA (1..1)}
 	{Decreto Legislativo (como Norma Jurídica)}
 	{D: (Rejeitado pela <ref>)}

  \relacionamento
  	{Medida Provisória (como Norma Jurídica)}
 	{REJEITADA (1..1)}
 	{Rejeição de Medida Provisória}
 	{(1..1) Rejeição de Medida Provisória (1..1)}
 	{REJEITA (1..1)}
 	{Ato Declaratório do CN, do SF ou da CD (como Norma Jurídica)}
 	{D: (Rejeitada pela <ref>)}

\end{tabelarelacionamento}


% --
\begin{tabelarelacionamento}{Relacionamentos
de Atos Internacionais}{tab-relacionamento-atos-internacionais}

  \relacionamento
  	{Ato Internacional (Norma Jurídica)}
 	{APROVADO (1..1)}
 	{Aprovação de Ato Internacional}
 	{(1..n) Aprovação de Ato Internacional (1.. n)}
 	{APROVADOR (1..1)}
 	{Decreto Legislativo}
 	{D: (Aprovado por <ref> )}

  \relacionamento
  	{Ato Internacional (Norma Jurídica)}
 	{PROMULGADO (1..1)}
 	{Promulgação de Ato Internacional}
 	{(1..n) Promulgação de Ato Internacional (1.. n)}
 	{PROMULGADOR (1..1)}
 	{Decreto}
 	{D: (Promulgado por <ref>)}

\end{tabelarelacionamento}
